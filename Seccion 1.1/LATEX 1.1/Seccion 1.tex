
\documentclass[25pt]{article}

\usepackage{geometry}
\geometry{a4paper}
\setlength{\lineskip}{3.5pt}
\setlength{\lineskiplimit}{200pt}
\setlength{\baselineskip}{100pt}

\begin{document}
\linespread{1.25}
\begin{center}
 	{\Huge Proyecto - Sección 1}\\
	 \vspace{0.75cm}
	 {\large Probabilidad y Estadística ejercicios}\\
	 \vspace{0.5cm}
 	{\large Daniela Jijón, Juan Francisco Cisneros y Luciana Valdivieso}\\
	\vspace{0.5cm}
 	{\large 11 de junio de 2022}\\
 	\line(1,0){350}\\
	\end{center}
	
\section*{Tema}

El tema que escogimos para el proyecto es “Que estudiantes de la USFQ tiene mejor GPA, ¿los hombres o las mujeres?”, el tema lo desarrollamos en base a las recomendaciones sugeridas en la descripción del proyecto y nuestro interés por saber si existe influencia entre el sexo y el desempeño académico de los estudiantes de la USFQ. Al responder esta pregunta esperamos encontrar relaciones entre varios factores que pueden influir en el desempeño académico de los estudiantes.\par
\vspace{0.25cm}
Decidimos utilizar un total de 6 variables, 4 variables cuantitativas y 2 cualitativas para el proyecto, donde constan datos de cada estudiante como el sexo, colegio académico al que pertenece dentro de la USFQ, edad, el año que está cursando dentro de la universidad, cuántas horas de estudio a la semana dedica el encuestado y finalmente el GPA ponderado que el estudiante tiene actualmente.\par

\section*{Recolección de datos}

Para el proceso de recolección de datos utilizamos Microsoft Forms. Decidimos utilizar esta herramienta para poder crear y difundir una encuesta con mayor facilidad en redes sociales. Para poder descartar datos mal ingresados o falta de datos, hemos cambiado opciones dentro del software Microsoft Forms para que los usuarios tengan obligatoriamente que ingresar todos los datos de la encuesta.
El link de la encuesta se compartió por grupos de WhatsApp y en historias de Instagram, logrando obtener 202 observaciones en total, de las cuales cerca de 110 fueron obtenidas por la encuesta difundida de en redes. Para completar el resto de las observaciones generamos un código QR con enlace a la encuesta, nos acercamos personalmente a estudiantes de la universidad en el campus a pedirles que escaneen el código y llenen la encuesta. \par
\vspace{0.25cm}
Los problemas que enfrentamos fueron principalmente relacionados a la privacidad de datos de los encuestados y facilidad de acceso a la encuesta. Hubo incomodidad para los encuestados al tener que ingresar a las encuestas registrándose con el correo institucional, ya que les tomaba más tiempo, sin embargo, pudimos resolver este problema deshabilitando esta opción para que los encuestados tomen menos tiempo en responder. Para poder verificar la identidad de los encuestados les pedimos que proporcionen su número de celular, no obstante, esto se volvió un problema para algunas personas, ya que preferían mantener sus datos privados. De igual forma, hubo inconvenientes con la pregunta “¿Cuál es tu GPA ponderado?, debido a que algunos encuestados no sabían dónde ver esta información o no sabían distinguir el tipo de GPA que se les pedía. \par
\vspace{0.25cm}
Por consiguiente, entre las recomendaciones que daríamos a otras personas que quisieran recoger la misma información está siempre mantener la información de los encuestados anónima, especialmente si el tema es bastante personal como el que elegimos. De igual manera, recomendamos evitar el uso de palabras confusas para los encuestados y explicar más detalladamente a lo que se refiere la pregunta, por ejemplo, en este caso proporcionar un link en la descripción de la encuesta que lleve a los estudiantes al Banner, donde estos puedan revisar el GPA que se pide que ingresen en la encuesta.  \par
\vspace{0.5cm}

{\bfseries\underline {Variables cualitativas}}
\begin{description}
\item[Sexo]    Las opciones de respuesta fueron (Masculino, Femenino, Otro), los datos obtenidos son (102, 99, 1) respectivamente.
\item[Colegio académico] Los estudiantes podían elegir entre los colegios\\ (CADI, COCIBA, CADE, COCOA, COCSA, CHAT, JUR, COM, POLI, COCISOH), y obtuvimos los siguientes resultados (8, 11, 26, 27, 29, 5, 7, 2, 73, 14) respectivamente, la mayoría de encuestados provienen del politécnico.
\end{description}
\vspace{0.50cm}
{\bfseries\underline {Variables cuantitativas}}
\begin{description}
\item[Edad]    Los estudiantes tenian como única opción ingresar una variable numérica, obtuvimos datos de estudiantes con edades entre 18 y 33 años
\item[Año]  Los datos varían entre el primer año (1) y sexto año (6), hemos obtenido (1 = 55, 2 = 67, 3= 50, 4 = 19, 5 = 9, 6= 2)
\item[Horas de estudio]  Las horas de estudio que el estudiante dedica por semana, se han obtenido datos por clases, comenzando desde 0 horas hasta más de 25 horas. Los datos obtenidos fueron ([0-5] = 36, [6-10] = 61, [11-15] = 34, [16-20] = 43, [21-25] = 13, [25 en adelante) = 15)
\item[GPA]  Para esta variable hicimos uso de una herramienta de Microsoft Forms para que los datos ingresados solo sean numéricos y entre (0,4). Obtuvimos GPAs ponderados de mínimo 2 y máximo 4. 
\end{description}

\end{document}